%---DOCUMENT-------------------------------------------------------------------

\documentclass[a4paper,10pt]{report}
\usepackage[french]{babel}
\usepackage[T1]{fontenc}
\usepackage{listings}

%---PACKAGES-------------------------------------------------------------------

\usepackage{makeidx} \makeindex
\usepackage[Rejne]{fncychap}				% Lenny, Conny ,Bjarne, Rejne, Glenn, Sonny
\usepackage{fancyhdr}
\usepackage{eurosym}
\usepackage{lastpage}
\usepackage{a4wide}
\usepackage[french]{minitoc}
\usepackage[hmargin=1cm,vmargin=2cm]{geometry}

%---SORTIES--------------------------------------------------------------------

\newif\ifpdf

\ifx\pdfoutput\undefined
   \pdffalse
\else
   \ifnum\pdfoutput=0
      \pdffalse
   \else
      \pdfoutput=1 \pdftrue
   \fi
\fi


%---PDF------------------------------------------------------------------------

\ifpdf
\usepackage[pdftex]{graphicx, color}
\graphicspath{{images/}}
\DeclareGraphicsExtensions{.jpg,.png}
\pdfcompresslevel=9

\usepackage[pdftex, 					% Param�trage de la navigation
bookmarks = true, 						% Signets
bookmarksnumbered = true, 		% Signets num�rot�s
pdfpagemode = None, 					% None, UseThumbs, UseOutlines, Fullscreen
pdfstartview = FitH, 					% FitH, FitV, FitR, FitB, FitBH, FitBV, Fit
pdfpagelayout = OneColumn, 		% SinglePage, OneColumn, TwoColumnLeft, TwoColumnRight
colorlinks = false, 					% Liens en couleur
urlcolor = black, 						% Couleur des liens externes
pdfborder = {0 0 0} 					% Style de bordure : ici, rien
]{hyperref}

\hypersetup{
pdfauthor = {\textsc{Huge Software}\\ Thomas A\"it-Taleb, Dimitri Georgoulis, Pierre Guilbert et Alexandre Testu}, 							% Auteurs
pdftitle = {Rapport de soutenance}, 								% Titre du document
pdfsubject = {Deuxi\`eme Soutenance}, 							% Sujet
pdfkeywords = {}, 						% Mots-clefs
pdfcreator = {}, 							% Logiciel qui a cr�e le document
pdfproducer = {} 							% Soci�t� avec produit le logiciel
plainpages = false}
\usepackage{pdfpages}

%---DVI------------------------------------------------------------------------

\else
\usepackage{graphicx}
\graphicspath{{eps/}}
\newcommand{\url}[1]{\emph{#1}}
\newcommand{\href}[2]{\emph{#2}[1]}
\fi

%---EN-TETE-ET-PIED-DE-PAGE----------------------------------------------------

\renewcommand{\headrulewidth}{0.5pt}
\renewcommand{\footrulewidth}{0.5pt}
\pagestyle{fancy}

%\lhead{}
%\chead{}
%\rhead{}
%\lfoot{}
%\cfoot{}
%\rfoot{}

%---PAGE-DE-GARDE--------------------------------------------------------------

\title{\textsc{Rapport de Soutenance} \\ Deuxi\`eme Soutenance}
\author{\textsc{Huge Software}\\ \\ Thomas A\"it-Taleb \\ Dimitri Georgoulis \\ Pierre Guilbert \\ Alexandre Testu}
\date{}

%---COLOR---------------------------------------------------------------------

%\pagecolor{}
% \color{}

%---DEBUT-DU-DOCUMENT----------------------------------------------------------

\begin{document}
\lstset{language=C}
\dominitoc
\maketitle
\tableofcontents \pagebreak
\thispagestyle{fancy}

\chapter{Introduction} % (fold)
\label{cha:introduction}



% chapter introduction (end)


\chapter{Interface Utilisateur} % (fold)
\label{cha:interface_utilisateur}

	\section{Enregistrement d'un fichier texte} % (fold)
	\label{sec:enregistrement_d_un_fichier_texte}
	
Lors de la premi\`ere soutenance, nous pouvions ouvrir une image (\'enorme, hein?). C'est \`a peu pr\`es tout.
Beaucoup de choses ont chang\'e depuis le mois de F\'evrier.
Les modifications apport\'ees sont relatives au texte. Nous pouvons d\'esormais \'editer du texte, ainsi que l'enregistrer. Soyons franc: la gestion du texte sous Gtk, c'est pas de la tarte. On utilise un widget de type GtkTextView qui nous sert \`a afficher le texte. Pour l'\'editer, on utilise un GtkTextBuffer, qui utilise des GtkTextIter associ\'es \`a des GtkTextMarks. Bref, c'est un merdier sans nom.
Les GtkTextIter servent \`a se rep\'erer dans le GtkTextBuffer (qui est un buffer, thank you captain Obvious!). Un GtkTextIter est un endroit du texte. On l'utilise par exemple pour r\'ecup\'erer une cha\^ine de caract\`eres:
	\begin{lstlisting}
     text = gtk_text_buffer_get_text(txtbuffer), &iStart, &iEnd, FALSE );
	\end{lstlisting}
	Ici, iStart et iEnd sont des pointeurs sur des GtkTextIter (ici le d\'ebut et la fin du buffer, encore une fois, thank you captain obvious!).
	Il faut ensuite cr\'eer une procedure pour sauvegarder le texte r\'ecup\'er\'e. En C, on utilise le type \verb!FILE! de la biblioth\`eque \verb!stdio.h!. C'est assez simple pour ce qu'on fait ici. On ouvre un fichier puis on \'ecrit dedans, sans oublier une gestion d'erreur (ici pas forc\'ement tr\`es utile, mais il n'est jamais inutile de g\'erer les erreurs\ldots).
	\begin{lstlisting}
    /* text est le texte � enregistrer et filename le chemin du fichier */
  void save_as (char *text, char *filename)
  {  
    FILE *fp;                                
      /* on ouvre le fichier en mode �criture ("w")*/
    fp = fopen(filename, "w"); 
    if(!fp)                                
        /* on renvoie un message d'erreur */  
      fprintf(stderr, "Can't open file\n");
    else       
        /* on �crit "text" dans le fichier */                              
      fprintf(fp, text); 
      /* on ferme le fichier */
    fclose(fp);          
  }
	\end{lstlisting}
	% section enregistrement_d_un_fichier_texte (end)
	
	\section{Ortaugraffe} % (fold)
	\label{sec:ortaugraffe}
		L'utilisateur aura besoin d'un correcteur orthographique non seulement pour corriger les erreurs de notre OCR mais aussi pour \'editer son texte avant de l'enregistrer. Avant de me lancer dans la cr\'eation d'un syst\`eme de correction orthographique, je voyais d\'ej\`a le tableau: des biblioth\`eques compliqu\'ees, des centaines de fonctions qui font tout sauf ce qu'on veut, etc\ldots J'avais en effet fini par devenir tr\`es pessimiste vis-\'a-vis de Gtk. Je dois dire que j'\'etais bluff\'e. Une ligne. UNE ligne. On utilise la biblioth\`eque \verb!GtkSpell! pr\'esente sur les machines du PIE. Cette fameuse ligne, la voici dans tout sa splendeur:
		\begin{lstlisting}
    gtkspell_new_attach (GtkTextView *view,
                         const gchar *lang,
                         GError **error);
		\end{lstlisting}
		\verb!view! est notre widget \verb!GtkTextView!, \verb!lang! est la langue, \verb!error! est un pointeur sur la localisation d'une erreur \'eventuelle (\verb!NULL! pour nous).\\
		C'est gr\^ace \`a cette ligne magique que l'on obtient cette merveille:
		
		%%% SCREEN CORREC
		
	% section ortaugraffe (end)
% chapter interface_utilisateur (end)


\end{document}

%---FIN-DE-DOCUMENT------------------------------------------------------------
