%---DOCUMENT-------------------------------------------------------------------

\documentclass[a4paper,10pt]{report}
\usepackage[french]{babel}
\usepackage[utf8]{inputenc}
\usepackage[T1]{fontenc}
\usepackage{listings}

%---PACKAGES-------------------------------------------------------------------

\usepackage{makeidx} \makeindex
\usepackage[Rejne]{fncychap}				% Lenny, Conny ,Bjarne, Rejne, Glenn, Sonny
\usepackage{fancyhdr}
\usepackage{eurosym}
\usepackage{lastpage}
\usepackage{a4wide}
\usepackage[french]{minitoc}
\usepackage[hmargin=3.5cm,vmargin=2cm]{geometry}

%---SORTIES--------------------------------------------------------------------

\newif\ifpdf

\ifx\pdfoutput\undefined
   \pdffalse
\else
   \ifnum\pdfoutput=0
      \pdffalse
   \else
      \pdfoutput=1 \pdftrue
   \fi
\fi


%---PDF------------------------------------------------------------------------

\ifpdf
\usepackage[pdftex]{graphicx, color}
\graphicspath{{images/}}
\DeclareGraphicsExtensions{.jpg,.png,.gif}
\pdfcompresslevel=9

\usepackage[pdftex, 					% Paramétrage de la navigation
bookmarks = true, 						% Signets
bookmarksnumbered = true, 		% Signets numérotés
pdfpagemode = None, 					% None, UseThumbs, UseOutlines, Fullscreen
pdfstartview = FitH, 					% FitH, FitV, FitR, FitB, FitBH, FitBV, Fit
pdfpagelayout = OneColumn, 		% SinglePage, OneColumn, TwoColumnLeft, TwoColumnRight
colorlinks = false, 					% Liens en couleur
urlcolor = black, 						% Couleur des liens externes
pdfborder = {0 0 0} 					% Style de bordure : ici, rien
]{hyperref}

\hypersetup{
pdfauthor = {\textsc{Huge Software}\\ Thomas A\"it-Taleb, Dimitri Georgoulis, Pierre Guilbert et Alexandre Testu}, 							% Auteurs
pdftitle = {Rapport de soutenance}, 								% Titre du document
pdfsubject = {Deuxi\`eme Soutenance}, 							% Sujet
pdfkeywords = {}, 						% Mots-clefs
pdfcreator = {}, 							% Logiciel qui a crée le document
pdfproducer = {} 							% Société avec produit le logiciel
plainpages = false}
\usepackage{pdfpages}

%---DVI------------------------------------------------------------------------

\else
\usepackage{graphicx}
\graphicspath{{eps/}}
\newcommand{\url}[1]{\emph{#1}}
\newcommand{\href}[2]{\emph{#2}[1]}
\fi

%---EN-TETE-ET-PIED-DE-PAGE----------------------------------------------------

\renewcommand{\headrulewidth}{0.5pt}
\renewcommand{\footrulewidth}{0.5pt}
\pagestyle{fancy}

%\lhead{}
%\chead{}
%\rhead{}
%\lfoot{}
%\cfoot{}
%\rfoot{}

%---PAGE-DE-GARDE--------------------------------------------------------------

\title{\textsc{Rapport de Soutenance} \\ Deuxi\`eme Soutenance}
\author{\textsc{Huge Software}\\ \\ Thomas A\"it-Taleb \\ Dimitri Georgoulis \\ Pierre Guilbert \\ Alexandre Testu}
\date{}

%---COLOR---------------------------------------------------------------------

%\pagecolor{}
% \color{}

%---DEBUT-DU-DOCUMENT----------------------------------------------------------



\begin{document}
\lstset{language=C}
\dominitoc
\maketitle
\tableofcontents \pagebreak
\thispagestyle{fancy}

\chapter{Introduction} % (fold)
\label{cha:introduction}

Ce rapport pr\'esente l'avancement du projet OCRe durant les six semaines qui ont suivies la premi\`re soutenance. Nous avons beaucoup de nouveaut\'es \`a vous montrer, du r\'eseau de neurones \`a l'interface graphique, sans oublier le pr\'etraitement de l'image ou encore la segmentation. M\^eme le site Web a eu sa dose de changements. Nous avons essay\'e de distribuer le travail de fa\c con \'egale entre les 4 membres du groupe. Il faut pourtant garder \`a l'esprit que nous avons tous \`a un moment ou \`a un autre aid\'e un autre membre, parce qu'apr\`es tout c'est l\`a tout l'int\'er\^et de travailler en groupe!

Dimitri s'est consacr\'e au r\'eseau de neurones, tandis que Pierre a continu\'e son travail d\'ebut\'e lors de la premi\`ere soutenance sur le pr\'etraitement de l'image. Thomas, lui s'est occup\'e de la segmentation et Alexandre est toujours charg\'e de l'interface graphique et du site Web.


% chapter introduction (end)

\chapter{Analyse et extraction}
\label{cha:analyse_et_extraction}
Lors de la première soutenance, nous vous avions présenté un algorithme
de détection des lignes. Celui-ci avait pour but d'extraire les lignes
de caractères d'àprès une image parfaite, à savoir une image prétraitée.

Malgré un fonctionnement correct sur des images simples, cette
algorithme ne prenait pas en compte les lettrines, ni les colonnes de
texte par exemple. En effet utilisant la méthode des projections
horizontales, après analyse de l'histogramme obtenu, on pouvait
distinguer convenablement les pics représentant les lignes ainsi que les
creux pour les interlignes. Cependant, cette histogramme ne nous
fournissait aucune information significative en ce qui concerne les
lettrines et les colonnes multiples.

Notre idée de départ pour cette soutenance était d'étendre cette méthode
en projetant verticalement pour chaque ligne extraite afin d'obtenir
chacun des caractères. Cependant, nous nous sommes rendus compte que
débuter par une détection des blocs serait une meilleure solution pour
pouvoir envisager par la suite une détection efficace des lignes et des
caractères. C'est pourquoi l'algorithme a été totallement remanié pour
constituer ainsi une approche montante de l'extraction des éléments.

\section{Principe}
L'algorithme développé pour cette soutenance est basé sur le principe de
la segmentation par régions (region-based segmentation). 
Il consiste à regrouper des pixels semblables dans le but de créer des
zones homogènes: les régions. Pour ce type de
segmentation, on distingue deux principes: la décomposition/fusion
(split/merge) et la croissance de régions (growing-region). Nous avons
choisi la méthode de croissance de régions qui nous paraissait plus
adaptée au type d'images à traiter. Elle se décompose en deux étapes
principales:

  \begin{enumerate}
    \item Trouver les points de départ de nos régions.
    \item Faire grossir nos régions par agglomération des pixels
    voisins.
  \end{enumerate}

Le principe général est de trouver l'ensemble des blocs de pixels noirs
contigus, que nous appelerons par la suite composantes connexes, et de
les regrouper ensuite en blocs, représentant alors les
lignes et les paragraphes. \\
Notre algorithme se basera essentiellement sur les deux types d'éléments
principaux que sont les composantes connexes (nos points de départ) et
les blocs (nos régions). Ces deux types sont représentés par des
structures définies comme ceci:

\begin{lstlisting}
struct s_cc_elt
{
 int id;
 int nbpix;
 struct s_cc_coordinate coord;
 short int chr; /* if >0 then cc is a character */
 struct s_cc_elt *next;
};

struct s_block_elt
{
 int id;
 struct s_cc_list *cclist;
 int nbcc;
 int posx; /* Upper left hand corner x coordinate */
 int posy; /* Upper left hand corner y coordinate */
 int height;
 int width;
 int type; /* UNKNOW (0), TEXT (1), IMAGE (2) */
};
\end{lstlisting}

Notre structure \verb!s_cc_elt! représente une composante connexe ainsi que
ses caractéristiques telles que sa position dans l'image, le nombre de
pixels qu'elle contient, son identifiant, un indicateur pour savoir si
nous avons à faire à un caractère et bien entendu un lien vers l'élément
suivant. En effet, notre algorithme utilise des listes chainées pour
représenter l'ensemble des composantes connexes ainsi que celui des
blocs.
La structure \verb!s_block_elt! représente comme son nom l'indique les blocs et
ses caractéristiques qui sont constituées de son identifiant, le nombre
de composantes connexes englobées dans le bloc, les données permettant
de le situer spatialement dans la matrice binaire (représentant
l'image) et son type.

Afin de mieux organiser notre code, sa compréhension et son débuggage,
noua avons décomposé le problème de la segmentation en plusieurs
sous-problèmes. Ceci nous a donc amené à faire de même avec l'algorithme
que nous avons préalablement découpé en plusieurs fonctions.

\subsection{Détection des composantes connexes}
Premièrement, notre fonction principale prend en paramètre une matrice
binaire,représentant les pixels noirs et blancs d'une image prétraitée, et
renvoie une liste de composantes connexes. Voici son prototype:

\begin{lstlisting}
t_cc_list *findCC(t_matrix *matrix);
\end{lstlisting}

Elle consiste à réaliser tout d'abord une matrice de marque afin
d'éviter de parcourir plusieurs fois un pixel, puis elle parcoure notre
matrice binaire à la recherche des pixels noirs (à savoir ici des 1).
Dès qu'elle en a trouvé un, elle fait appel à la fonction de création de
composante connexe dont voici le prototype simplifié:

\begin{lstlisting}
t_cc_list *makeCC(int cc_count,
                  t_matrix *matrix,
                  char **mark,
                  t_cc_list *cc_list);
\end{lstlisting}

Cette fonction permet comme indiqué de créer un élément composante
connexe. Elle prend en paramètre le numéro de la nouvelle composante
connexe, la matrice binaire, la matrice de marque. Le paramètre \verb!cc_list!
est un paramètre résultat dans lequel on mettra à jour la liste des
composante connexe. Cette fonction crée donc une composante connexe avec
ses caractéristiques et met à jour la liste. Entre temps, elle fait
appel à la fonction permettant de parcourir les pixels noirs constituant
la composante connexe dont le prototype simplifié est:

\begin{lstlisting}
void crossCC(t_cc_elt *elt,
             t_matrix *matrix,
             char **mark);
\end{lstlisting}

Cette fonction est utilisée pour déterminer l'ensemble de l'étendue
composante connexe. Elle permet alors de mettre à jour le nombre de
pixels la composant ainsi que ses coordonnées minimum et maximum qui
permettront par la suite de retrouver la matrice correspondant à un
caractère.
Cette fonction est un point clé de la segmentation dans la mesure où
c'est elle qui est responsable de déterminer si oui ou non tel ou tel
pixel doit être inséré dans la composante.
Son principe se base sur un parcours de la matrice de voisinage du pixel
courant à la recherche des pixels noirs. Dès que l'on en trouve un, on
place dans une file. A la fin du parcours, on refait le traitement sur
le pixel défilé tant que la file n'est pas vide.

% SEG IMAGES

\subsection{Analyse des composantes connexes}
Une fois la liste des composantes connexes obtenues, il convient
d'analyser chacune d'entre elles afin de voir si elles correspondent à
un caractère ou non.

\begin{lstlisting}
void checkIfCharacter(t_cc_list *cc_list)
\end{lstlisting}

La fonction responsable de ce traitement (prototype simplifié ci-dessus)
prend en paramètre la liste des composantes connexes et la met à jour.
Elle effectue des mesures de seuil et détermine si une composante
connexe est un caractère.

% CARACTERES IMAGES


\subsection{Création des blocs}
Nous arrivons maintenant à la phase 2) de la segmentation par régions.
Cette phase consiste à regrouper les composantes connexes ensemble afin
de former des blocs qui correspondront aux lignes puis aux paragraphes.
Le prototype de la fonction est le suivant:

\begin{lstlisting}
t_block_list *makeBlocks(t_cc_list *cc_list)
\end{lstlisting}

Cette fonction prend en paramètre la liste des composantes connexes et
renvoie une liste de blocs. Nous avons commencé à implémenter cette
fonction, cependant n'ayant pas encore réussi à arriver à un niveau
convenable et fonctionnel, aucune présentation à ce sujet ne sera fait
à la soutenance. On accuse donc un retard en ce qui concerne le
regroupement par blocs.

\section{Exécutable}
Afin de concrétiser cette méthode de segmentation, nous avons développé
un exécutable fonctionnel, nommé \verb!seg! permettant
d'extraire les caractères d'une image d'entrée et d'enregister un
fichier de sortie présentant la segmentation par caractères. Cette
exécutable est utilisable avec plusieurs arguments dont voici la
description:

\begin{verbatim}
Usage: seg -i inputfile -o output file
 This program extracts the connected 
 components of a bitmap image
 It saves the extraction result in an ouput bitmap file
 Arguements:
    -i file: path of your input file
    -o file: path of your output file
    -v     : verbose
    --help : display the help
\end{verbatim}

Les arguments \verb!-i [File] -o [File]! sont obligatoires et permettent
de spécifier une image bitmap d'entrée et le nom du fichier de sortie.
Le programme est également muni de deux options. Tout d'abord,
\verb!--help! fait référence à une aide sucscinte. Enfin, l'option
\verb!-v! permet d'activer le mode verbose qui affiche la liste des
composantes connexes extraites ainsi que leurs caractéristiques.

\section{Prochaine soutenance}
Pour la prochaîne soutenance, nous devons rattraper le retard accumulé
sur la détection des blocs et pouvoir fournir une liste de blocs en
bonne et dû forme. Liste qui nous permettra entre autres par la suite de
reconnaître la structure de la page. Nous devrons trouver un format
adéquate pour stocker ces informations de structures de pages, le xml
serait une solution possible. Egalement, nous devrons interfacer la
segmentation avec d'un côté le prétraitement et de l'autre côté le
réseau de neurones. En ce qui concerne le côté interface
segmentation/réseau de neurones, nous devrons standardiser les données
pour l'entrée du réseau. Il semblerait qu'une vectorisation ou bien
l'utilisation des chain codes (Freeman) soit une éventualité.










%%debut chapitre sur le programme ocred
\chapter{ OCRed, k\'ezako ? }
 OCRed est n\'e des mots Ocre et digital, qui rappellons le encore une fois
 signifient ``Optical Character Recognition enhanced'' pour l'un et
 num\'erique pour l'autre. La partie du projet portant sur le
 pr\'etraitement de l'image devant \^etre faite en ocaml, OCRed est
 totalement r\'ealis\'e dans ce langage.

\section{ Visualisation de l'image grace a l'executable }
 La cr\'eation du logiciel de traitement de l'image nous a men\'e, au
 d\'ebut du projet \'a vouloir visualiser directement, c'est \'a dire
 sans logiciel tiers, l'action de notre travail.
 La visualisation des images trait\'ees permet notre ind\'ependance
 au niveau de l'utilisation du logiciel, ce qui est un confort certain.
 La fen\^etre est r\'ealis\'ee avec OcamlSDL, tout comme le pr\'etraitement
 de l'image.
%% mettre image de la fenetre sdl

\begin{center}
	
	\includegraphics[width=70mm]{visualisationimage.jpg}\\
\end{center}

\section{ Utilisation de l'\'executable }
 L'ex\'ecutable en ligne de commande admet un certain nombre d'arguments,
 certains ne pouvant \'evidemment donner lieu \`a un quelconque r\'esultat
 sans la sp\'ecification d'une image d'entr\'ee.
\subsection{ Sp\'ecifier une image d'entr\'ee }
 Il est plus logique, pour un logiciel de traitement d'image, d'avoir
 une image \`a traiter. Vous devez pour cela utiliser l'argument ``-i''
 suivi du chemin d'une image de type jpeg,bmp,png au moins.
\subsection{ Sp\'ecifier une image de sortie }
 Vous n'\^etes pas oblig\'e de pr\'eciser le chemin de l'image de sortie
 mais cela permet une certaine flexibilit\'e. Dans le cas o\`u vous
 n'indiquez pas la sortie, OCRed se debrouille pour sortir une image
 avec un nom assez \'evident, d\'ependant du traitement. Par exemple
 rotatation.bmp ou encore tresholded.bmp, les images se trouveront l\`a
 o\`u vous avez ex\'ecut\'e votre programme. Pour sp\'ecifier un chemin de
 sortie, veuillez utiliser l'argument ``-o'' suivi du chemin d'une image
de type jpeg,bmp,png au moins.
\subsection{ La visualisation de l'image }
 Gr\^ace \`a OcamlSDL, nous pouvons visualiser notre image, il faut pour
 cela sp\'ecifier une image d'entr\'ee, et passer l'argument ``-d'' en
 param\`etre. Les d\'esavantages de l'utilisation de cette interface sont
 assez subtils, par exemple vous ne pouvez ni zoomer ni d\'ezoomer, ni
 vous d\'eplacer sur l'image. Toutefois il y'a (aussi) des avantages, vous pouvez
 visualiser directement les modifications appliqu\'ees \`a votre
 image. En appuyant sur F2 vous faites tourner l'image de l'angle que
 vous avez pass\'e en param\`etre --voir plus loin--. En appuyant sur F3
 vous appliquez le seuil, que vous pouvez sp\'ecifier en ligne de
 commande. En appuyant sur F4, vous appliquez le filtre m\'edian. Nous
 tenons \`a vous rappeler que le seuil ne marche que sur les couleurs.
 Pour quitter il suffit d'appuyer sur F1, la petite croix en haut \`a
 droite ne marche pas.
\subsection{ Le seuil }
 Permet le passage d'une image en noir et blanc, pour avoir un minimum
 de pertes d'informations vous devez sp\'ecifier un seuil viable. Par
 d\'efaut le seuil est \`a 200, ce qui est suffisant pour un contraste
 normal de l'image. Vous devrez l'augmenter si il est trop faible et le
 baisser dans le cas contraire. Passer l'argument ``-s'' en
 param\`etre si vous voulez changer la valeur par d\'efaut. Pour
 l'anecdote ``-s'' viendrait de seuil, mais nous n'en sommes pas vraiment
 s\^urs.
\subsection{ Rotation }
 Vous pouvez passer un angle en pr\'ecision r\'eelle, ou enti\`ere. Pour
 cela passez respectivement l'angle avec l'argument ``-af'' et ``-a''.
 Rappelons que l'angle pass\'e en param\`etre est en degr\'es et non en
 radian. Vous pouvez aussi tourner l'image de 90 degr\'es dans le sens
 direct ou indirect, gr\^ace aux options ``-right'' et ``-left''.
\subsection{ Redimensionnement }
 Le redimensionnement peut s'effectuer en pourcentage de l'image avec
 l'option ``--resizepercent'' qui prend en param\`etre un pourcentage
 entier. L'on peut aussi redimensionner l'image en valeur absolue
 --c'est \`a dire en sp\'ecifiant des valeurs de hauteur et de largeur
 arbitraires--. Alexandre Testu avait besoin d'un petit redimensionnement
 rien que pour lui, c'est pour cela que l'option ``--resize-auto''
 existe.
\subsection{ D\'etection d'angle}
 Il existe un argument magique ``-dev'' qui a juste besoin d'avoir une
 image qui n'est pas bien droite en param\`etre. Il retourne directement
 l'image dans le bon sens, et en noir et blanc.
\subsection{ Help me if you can! }
 L'option ``-help'' est aussi pr\'esente afin d'accompagner
 l'utilisateur dans sa d\'ecouverte du programme. Pour tout autre
 interrogation, il se peut que vous puissiez trouver des informations
 plus ou moins utiles dans les sources du projet, au niveau du fichier
 README.
%%fin chapitre sur le programme ocred


%%debut chapitre detection de l'angle
\chapter{ La d\'etection de l'angle }
 La d\'etection de l'angle \'etait \`a faire pour la premi\`ere
 soutenance, malheureusement par un manque de temps certain elle ne f\^ut
 pas pr\^ete \`a temps. Nous arrivons \`a la deuxi\`eme soutenance avec
 deux types de d\'etection de l'angle de rotation du texte d'une image,
 ce qui a pour but de redresser l'image, afin d'avoir une segmentation
 \`a peu pr\`es correcte.

\section{ Diff\'erentes approches}
 La d\'etection de l'angle de rotation d'un texte dans une image est un
 probl\`eme assez complexe. En effet une image contenant du texte n'est
 pas compos\'ee de composantes continues, il est donc plus difficile d'isoler
 les caract\'eristiques des \'el\'ements. Deux principes majeurs nous
 sont apparus. Tout d'abord l'on \'etudie les caract\'eristiques d'une
 image pour chaque angle, on isole ensuite le bon angle ; ce
 processus est long et d'autre part encore assez impr\'ecis.
 L'autre principe \'etait de d\'etecter la pente d'une ligne de caract\`eres.
\subsection{ La multiple rotation }
 Dans cet algorithme, l'on reduit la taille de l'image pour gagner en
 temps d'ex\'ecution, le redimensionnement est calcul\'e en fonction de la
 taille d'origine de l'image, on suppose avoir du format A4 \`a chaque
 fois. On proc\`ede comme suit: On tourne de fa\c con dichotomique
 l'image, c'est \`a dire en prenant des angles plus ou moins
 grand. Jusqu'a trouv\'e un histogramme convenable. L'histogramme est
 produit \`a partir de la projection de tout les pixels de l'image sur
 l'axe des horizontales.

Preuve exp\'erimentale produite \`a partir d' une image multifontes,
multicouleurs, mulicolonnes:

 Comme vous pouvez le voir sur ces images, il est \'evident qu'il y'a une
 nette diff\'erence entre des images avec un texte mal orient\'e et un
 texte orthonormal. (mais o\`u est l'orthonormalit\'e dans ce monde ?)

%% histogramme5degree.jpg
%% commentaire de l'image

\begin{center}
	
	\includegraphics[width=120mm]{histogramme5degre.jpg}\\
\end{center}


Sur cette image on peut voir un histogramme d'une image avec un
texte ayant subi une rotation de 5 degr\'es. Il est clair que le nombre
de zone ou l'histogramme atteint 0 en ordonn\'ee est beaucoup plus faible
que les autres images (environ 8).


%% rotation1deghisto.jpg
%% commentaire de l'image
\begin{center}
	
	\includegraphics[width=120mm]{rotation1deghisto.jpg}\\
\end{center}

Sur cette image on peut voir un histogramme d'une image avec un texte
ayant subi une rotation de 1 degr\'e. Par rapport \`a l'image
pr\'ec\'edente, on voit de fa\c con \'evidente que le nombre d'ordonn\'ees
\`a z\'ero est bien plus grand avec un angle de un degr\'e (ici environ 18).

%% histogramme_bon.jpg
%% commentaire de l'image

\begin{center}
	
	\includegraphics[width=120mm]{histogramme_bon.jpg}\\
\end{center}

Sur cette image on peut voir un histogramme d'une image avec un texte
n'ayant pas subi de rotation. Il est tr\'es net que le nombre
d'ordonn\'ees \`a z\'ero est le plus \'elev\'e.

Il est facile d'implementer un algo qui compte le nombre de bandes
proche de z\'ero ; il s'av\`ere que nous avons voulu compter le nombre de
sommets, ce qui est beaucoup moins pr\'ecis. La m\'ethode expos\'ee
ci-dessus semble bien marcher en th\'eorie.

\subsection{ Detection de la pente }
On d\'etecte la pente d'une ligne de texte ; pour cela on parcourt
l'image pour chaque ligne de pixels jusqu'\`a ce que l'on tombe sur un
pixel noir. Ceci permet d'avoir une certaine id\'ee de la pente d'une ligne
de texte. La tangente de cette pente nous donne l'angle par rapport \`a
la verticale. On la soustrait \`a 90 degr\'es si elle est positive, on
l'ajoute sinon. On se retrouve avec un angle qui correspond, avec un
taux d'erreur proche du dixi\'eme de degr\'e d'angle.

% mettre l'image histo_pente.jpg
% commentaire image histo_pente

\begin{center}
	
	\includegraphics[width=120mm]{histo_pente.jpg}\\
\end{center}

Les extrema locaux de la courbe sont beaucoup moins \'elev\'es en
terme de valeurs. Ce qui a pour cons\'equence, une pente peu \'elev\'ee et donc un angle tr\`es \'elev\'e.

% mettre l'image histo_pent_noangle.jpg
% commentaire image histo_pent_noangle


\begin{center}
	
	\includegraphics[width=120mm]{histo_pente_noangle.jpg}\\
\end{center}

Les extrema locaux de la courbe sont beaucoup plus \'elev\'es en
terme de valeurs que ceux de la courbe pr\'ec\'edente. Ce qui a pour
cons\'equence, une pente tres \'elev\'ee et donc un angle tr\'es faible.

Le probl\`eme avec cette technique, et le manque de pr\'ecision autour
de l'angle nul, et des bruits trop importants produits par des lignes,
tableaux, ou images impromptus.

Nous avons pourtant retenu cette technique pour cette soutenance, car
donnant lieu \`a des r\'esultats plus probants que la m\'ethode expos\'ee
plus haut.

\section{ Optimisations  }
 L'optimisation est un crit\`ere de confort pour l'utilisateur, mais
 surtout un gage de qualit\'e.
\subsection{ R\'eduction de l'image }
 Afin de pouvoir augmenter la rapidit\'e des algorithmes de d\'etection
 d'angle, nous r\'eduisons l'image de sorte qu'il y ait le moins de
 pertes d'informations n\'ecessaires \`a cette derni\`ere. A titre
 d'exemple une image au format A4 en 75 dpi, a une taille approximative
 de 900 par 700. Les caract\`eres pr\'esents dans cette image ont une
 taille en pixels d'\`a peu pr\`es 8 pixels. Il ne faut donc pas trop
 r\'eduire l'image, environ 40 pourcents de la taille originale. Au
 contraire pour une image de plus haute d\'efinition, par exemple 600
 dpi, il est n\'ecessaire de r\'eduire l'image \`a au moins 5 pourcents
 de la taille originale. Et comme dirait Junior, ``l'id\'ee c'est
 que'' la rotation d'une image r\'eduite \`a 5 pourcents prend environ
 0,05 secondes sur les machines du PIE, alors qu'elle prend plus de 30
 secondes pour une image en 600 dpi.

%% reduceimage20percent75dpit.jpg
\begin{center}
	
	\includegraphics[width=40mm]{reduceimage20percent75dpit.jpg}\\
	\caption{\emph{Une image r\'eduite.}}
\end{center}


\begin{description}
	\item[Remarque:] Il est int\'eressant de noter qu'une image r\'eduite comporte
	moins de bruit, qui fausserait l'avanc\'ee de l'algo de reconnaissance
	des plages nulles.
\end{description}

\subsection{ Ce qu'il nous reste \`a faire}
\subsubsection{ D\'etection d'angle}
Notre d\'etection de l'angle de rotation d'un texte dans une image est
encore assez impr\'ecise ; cela est principalement d\^u \`a l'utilisation
de l'arctangente qui devient tr\`es sensible aux extrema. Nous pensons
donc r\'eutiliser la technique de la multiple rotation, qui semble en
th\'eorie plus pr\'ecise. Les r\'esultats seront expos\'es \`a la
prochaine soutenance.
\subsubsection{ Optimisation des filtres}
Le gros probl\`eme de la rotation est la perte d'information, en effet
cette transformation a en sortie des nombres \`a virgule flottante,
ce qui ne correspond pas vraiment aux nombres entiers d\'efinissant la
grille d'une image dans le cas de pixels carr\'es. Une multitude de
filtres existent, mais le plus rapide et le plus adapt\'e dans le cas de la
rotation serait un filtre bicubique.

%% OCRed qu'est-ce que c'est pourquoi c'est faire.

%% = l'executable
%% = la ligne de commande
%%  les differentes options toussa


%% La detection de l'angle de rotation
%%  idees d'optimisation
%%  redimensionnement de l'image
%%  mettre des images
%%  histogramme et traitement du signal
%%  \mettre des images
%%  multiple rotation
%%  etudes de la pente.
%%  mettre des images
%% detection d'angle mettre des images avant apres

%% La rotation d'une image, une idee qui ne marche pas

%% Les idees qui n'ont pas vu le jour















\chapter{Interface Utilisateur} % (fold)
\label{cha:interface_utilisateur}

	\section{Enregistrement d'un fichier texte} % (fold)
	\label{sec:enregistrement_d_un_fichier_texte}

Lors de la premi\`ere soutenance, nous pouvions ouvrir une image (\'enorme, hein?). C'est \`a peu pr\`es tout.
Beaucoup de choses ont chang\'e depuis le mois de F\'evrier.
Les modifications apport\'ees sont relatives au texte. Nous pouvons d\'esormais \'editer du texte, ainsi que l'enregistrer. Soyons francs: la gestion du texte sous Gtk, c'est pas de la tarte. On utilise un widget de type GtkTextView qui nous sert \`a afficher le texte. Pour l'\'editer, on utilise un GtkTextBuffer, qui utilise des GtkTextIter associ\'es \`a des GtkTextMarks. Bref, c'est un merdier sans nom.
Les GtkTextIter servent \`a se rep\'erer dans le GtkTextBuffer (qui est un buffer, thank you captain Obvious!). Un GtkTextIter est un endroit du texte. On l'utilise par exemple pour r\'ecup\'erer une cha\^ine de caract\`eres:
	\begin{lstlisting}
   text = gtk_text_buffer_get_text( GTK_TEXT_BUFFER(txtbuffer), 
	                            &iStart, 
	                            &iEnd, 
	                            FALSE );
	\end{lstlisting}
	Ici, iStart et iEnd sont des pointeurs sur des GtkTextIter (ici le d\'ebut et la fin du buffer, encore une fois, thank you captain obvious!).
	Il faut ensuite cr\'eer une proc\'edure pour sauvegarder le texte r\'ecup\'er\'e. En C, on utilise le type \verb!FILE! de la biblioth\`eque \verb!stdio.h!. C'est assez simple pour ce qu'on fait ici. On ouvre un fichier puis on \'ecrit dedans, sans oublier une gestion d'erreur (ici pas forc\'ement tr\`es utile, mais il n'est jamais inutile de g\'erer les erreurs\ldots).
	\begin{lstlisting}
  void save_as (char *text, char *filename)
  {
    FILE *fp;
      /* opens the file in writing mode ("w")*/
    fp = fopen(filename, "w");
    if(!fp)
        /* sends an error message */
      fprintf(stderr, "Can't open file\n");
    else
        /* writes "text" in the file */
      fprintf(fp, text);
      /* closes the file */
    fclose(fp);
  }
	\end{lstlisting}
	% section enregistrement_d_un_fichier_texte (end)

	\begin{center}

		\includegraphics[width=120mm]{gui.png}\\
		\caption{\emph{Notre magnifique interface graphique}}\\
	\end{center}


	\section{Ortaugraffe} % (fold)
	\label{sec:ortaugraffe}
		L'utilisateur aura besoin d'un correcteur orthographique non seulement pour corriger les erreurs de notre OCR mais aussi pour \'editer son texte avant de l'enregistrer. Avant de me lancer dans la cr\'eation d'un syst\`eme de correction orthographique, je voyais d\'ej\`a le tableau: des biblioth\`eques compliqu\'ees, des centaines de fonctions qui font tout sauf ce qu'on veut, etc\ldots J'avais en effet fini par devenir tr\`es pessimiste vis-\'a-vis de Gtk. Je dois dire que j'ai \'et\'e bluff\'e. Une ligne. UNE ligne. On utilise la biblioth\`eque \verb!GtkSpell! pr\'esente sur les machines du PIE. Cette fameuse ligne, la voici dans tout sa splendeur:
		\begin{lstlisting}
    gtkspell_new_attach (GtkTextView *view,
                         const gchar *lang,
                         GError **error);
		\end{lstlisting}
		\verb!view! est notre widget \verb!GtkTextView!, \verb!lang! est la langue, \verb!error! est un pointeur sur la localisation d'une erreur \'eventuelle (\verb!NULL! pour nous).

		%%% SCREEN CORREC

	% section ortaugraffe (end)
% chapter interface_utilisateur (end)

% j'ai utilisé des guillemets de ce type ", si ils ne sont pas valides
% faudra penser à les corriger =)

\chapter{Reseau de neurones} % (fold)
\label{cha:reseau_de_neurones}

\section{Introduction} % (fold)
\label{subsec:introduction}
A l'issue de la premi\`ere soutenance, nos recherches sur les r\'eseaux
neuronaux nous avaient men\'es vers l'\'elaboration d'un perceptron
multicouches, celui-ci \'etant reconnu comme la star de la
reconnaissance de caract\`eres.

Le temps a pass\'e', \'ecrivait G\'erard de Narval, pensant \`a sa jeunesse
et toutes ces belles ann\'ees d\'esormais r\'evolues... En ce qui
concerne le r\'eseau de neurones, le temps a \'egalement pass\'e et
notre perceptron multicouches commence \`a prendre de l'allure. Sa
jeunesse n'\'etant pas encore r\'evolue, grand bien lui fasse, il a
encore beaucoup \`a apprendre... Alors
certes il ne reconna\^it pas encore grand chose, mais une premi\`ere
\'ebauche de sa structure est pr\^ete et les premiers tests ont \'et\'e
effectu\'es sur les donn\'ees d'un XOR et celles d'un afficheur sept
segments. La phase d'apprentissage reste \`a mettre au point car la
version actuelle ne donne pas les r\'esultats attendus. Mais ne parlons
pas de malheur, et rentrons d\`es \`a pr\'esent dans le vif du sujet : what about
multi-layer perceptron ?! 

Nous d\'ecrirons ici le perceptron tel que nous l'avons abord\'e, c'est \`a
dire \`a un stade encore exp\'erimental. Pour motiver sa conception,
nous nous sommes plac\'es dans le cas o\`u notre r\'eseau devra
reconna\^itre les caract\`eres provenant d'un afficheur sept
segments. En effet, les donn\'ees d'un tel alphabet sont assez simples
\`a formater sans pour autant atteindre le niveau de simplicit\'e d'une
fonction logique de base comme le OR ou l\'eg\`erement plus avanc\'ee
comme le XOR

% subsection introduction (end)
\pagebreak
\section{Structure et d\'efinitions} % (fold)
\label{subsec:structure_et_definitions}

Voici ce \`a quoi ressemble la b\^ete :
\begin{center}
	
	\includegraphics[width=120mm]{ra-perceptron.png}\\
	\caption{\emph{Perceptron \`a une couche cach\'ee}}\\
\end{center}

%pour ce paragraphe j'aimerais une liste à puce à la place de mes étoiles
%=)
Un perceptron est un ensemble de neurones reli\'es par des connexions
pond\'er\'ees. Ce n'est rien d'autre qu'un graphe non-orient\'e valu\'e
en somme, du moins en apparence. On y rep\`ere trois ensembles
particuliers de neurones que sont : 
%debut de liste à puce
*la couche d'entr\'ee : elle poss\`ede un certain nombre de neurones qui
correspondent \`aux entr\'ees du r\'eseau. La mod\'elisation et le
formatage des donn\'ees d'entr\'ee sont donc primordiaux si l'on veut
\'eviter d'avoir un trop grand nombre de neurones, ce qui augmenterait
le temps de la reconnaissance par la suite.

*la ou les couches cach\'ee(s) : ce sont ces neurones qui permettront de
reconna\^itre des donn\'ees. Le nombre de couches cach\'ees et de
neurones par couche est \`a adapter selon le probl\`eme et en
particulier selon la taille de l'\'echantillon de donn\'ees que l'on veut
apprendre au r\'eseau.

*la couche de sortie : ce sont ces neurones qui contiendront le
r\'esultat de la reconnaissance. Il y en a, a priori, autant que
d'\'elements \`a reconna\^itre, mais une seule sortie ensuite filtr\'ee
par une table de hachage,par exemple, reste envisageable.
%fin de liste a puce

% Mettre cette photo ici !

\begin{center}
	
	\includegraphics[width=70mm]{sortie008.png}\\
	\caption{\emph{Perceptron \`a une sortie}}\\
\end{center}




% avec la légende suivante : Perceptron \`a une sortie

La valeur d'un neurone est calcul\'ee \`a partir des valeurs de chacun
des neurones de la couche pr\'ec\'edente.

Les neurones artificiels ainsi \'evoqu\'es sont en fait tr\`es proches
des neurones naturels au niveau de leur fonctionnement. C'est entre
autres l'id\'ee d'int\'eractions avec une multitude d'autres neurones
qui permet au r\'eseau de se constituer une ``m\'emoire intelligente'',
apte \`a classifier ou reconna\^itre des donn\'ees qu'il n'a encore
``jamais'' rencontr\'ees auparavant (du moins pas exactement sous la
même forme) !

Avant de pouvoir utiliser un perceptron multicouches, il faut avant tout se mettre d'accord sur le
format de donn\'ees qu'il prendra en entr\'ee, puis effectuer un choix
judicieux de son nombre de couches cach\'ees et de son nombre de neurones
par couche cach\'ee. On doit ensuite ``entra\^iner'' le r\'eseau\ldots

% subsection structure_et_definitions (end)

\section{Formatage des donn\'ees} % (fold)
\label{subsec:formatage_des_donnees}

Pour que le r\'eseau reconnaisse, dans notre cas, si un caract\`ere est
telle ou telle lettre, il faut affecter une valeur pr\'ecise à chacun de
ses neurones d'entr\'e.
Parmi les id\'ees qui nous sont venues, il y a l'extraction de
caract\'eristiques des caract\`eres que l'on a isol\'e \`a partir du document
scann\'e. Ainsi l'on pourrait envisager de vectoriser chaque caract\`ere
afin d'en extraire des segments caract\'eristiques que l'on donnerait en
entr\'ee au r\'eseau. On pourrait aussi imaginer effectuer une premi\`ere
classification par crit\`ere de largeur, puis de hauteur, etc, en utilisant
plusieurs r\'eseaux en s\'erie par exemple, jusqu'\`a d\'etermination totale du
caract\`ere.

La m\'ethode que nous avons retenue \`a ce jour est n\'eanmoins bien
plus simpliste, et pas n\'ecessairement moins efficace : notre
perceptron recevra directement la matrice de pixels associ\'ee au
caract\`ere \`a reconna\^itre, apr\`es calibrage et centrage de chaque
caract\`ere dans une matrice de taille fixe.
Chaque neurone d'entr\'ee du r\'eseau recevra donc l'un des pixels de la
matrice de caract\`ere (d'o\`u la n\'ecessit\'e de r\'eduire au maximum
la taille et l'\'epaisseur de nos caract\`eres, afin de ne pas
d\'epasser la centaine de neurones d'entr\'ee).
Les donn\'ees d'apprentissage seront quant \`a elles pr\'esent\'ees sous
forme de couples de vecteurs d'entr\'ees/sorties au r\'eseau et
stock\'ees dans un fichier.


% subsection formatage_des_donnees (end)

\section{Param\`etres du r\'eseau} % (fold)
\label{subsec:choix_des_parametres_du_reseau}

Une fois les donn\'ees format\'ees et stock\'ees dans un fichier, il
faut choisir un certain nombre d'éléments clés du réseau.

Nous avons donc choisi d'affecter chaque pixel de la matrice du
caract\`ere \`a reconna\^itre \`a un neurone d'entr\'ee.

Dans le mod\`ele retenu actuellement, nous pr\'evoyons aussi d'affecter
un neurone de sortie par caract\`ere \`a reconna\^itre. Le caract\`ere
reconnu sera celui dont la valeur sera \`a un alors que les autres seront
rest\'es \`a z\'ero.

Pour ce qui est des couches cach\'ees, il est recommand\'e dans le cas
de la reconnaissance de caract\`eres d'en prendre entre une et deux, et
de choisir un nombre de neurones par couche du m\^eme ordre de grandeur
que le nombre de neurones de la couche de sortie. Mais nous essayerons
plusieurs configurations jusqu'\`a obtenir des r\'esultats
performants.

Un certain nombres de param\`etres relatifs \`a l'apprentissage des
donn\'ees par le r\'eseau doivent \'egalement \^etre fix\'es. Nous en
ferons une br\`eve description dans la section ``Entra\^inement du r\'eseau.''

% subsection choix_des_parametres_du_reseau (end)

\section{Entra\^inement du r\'eseau} % (fold)
\label{subsec:entrainement_du_reseau}

Cette partie est tr\`es probablement la plus int\'eressante de
l'impl\'ementation du perceptron multicouches. C'est en effet ici que
l'on doit mettre en oeuvre le fonctionnement du r\'eseau \`a proprement
parler. Nous l'abordons ici succintement, car il est écrit que l'on doit
se revoir une prochaine fois, et garder le meilleur pour la fin est,
après tout, une bonne règle de vie. 

Son principe est le suivant :

Dans un premier temps, on affecte des petites valeurs \`a tous les poids
des connexions neuronales (entre 0,5 et -0,5 dans notre cas).

Ensuite, pour un nombre d'it\'erations donn\'e par exemple, ou bien tant
que l'erreur totale du vecteur de sorties est supérieure à un seuil fixé, on effectue
les \'etapes suivantes :

On permute l'ordre des donn\'ees que l'on pr\'esente au r\'eseau, ceci
d'uniformiser l'apprentissage, de sorte \`a ce qu'il n'apprenne pas \`a reconna\^itre toujours la m\^eme
s\'equence de donn\'ees.

On fournit ensuite un couple d'entr\'ees/sorties d\'esir\'ees au
r\'eseau.

Il faut maintenant propager vers l'avant les valeurs des neurones
d'entr\'ee, de sorte \`a obtenir les valeurs de sortie.

Pour cela on utilise une fonction de combinaison, ici une somme
pond\'er\'ee sum, dans notre cas
d\'efinie comme, pour chaque neurone ni, la somme des produits des valeurs
des neurones nj qui pointent sur lui par les poids des connexions pij qui les
lient au neurone courant :

\[sum(ni) = somme\, sur\, j\, (pij(ni,nj)* valeur(nj))
\]% une formule en LaTeX serait plus jolie ici
\\

\begin{center}
	
	\includegraphics[width=70mm]{neuron.jpg}\\
	\caption{\emph{Calcul de la sortie d'un neurone}}\\
\end{center}




Cette valeur est ensuite pass\'ee en param\`etre d'une fonction
d'activation propre au perceptron, ici la fonction sigmo\"ide f avec :  

\[f(sum(ni)) = 1/(1+exp(-sum(ni))) 
\]% une formule en LaTeX serait plus jolie ici

On obtient ainsi une valeur exp\'erimentale pour chaque neurone de
sortie que l'on va pouvoir confronter \`a la valeur th\'eorique fournie
initialement.

On calcule donc l'erreur e observ\'ee pour chaque neurone de sortie ni avec :

\[
	e(ni) = (valeur\, theorique(ni)\, -\, valeur\, observee(ni))*(f'(sum(ni)))
\]

% une formule en LaTeX serait plus jolie ici

On propage ensuite l'erreur vers les couches intérieures 

Puis on modifie les poids de toutes les connexions du réseau.

Ces deux derniers points s'effectuent au moyen de l'algorithme de
r\'etropropagation du gradient, dont la structure est en fait
mod\'elis\'ee ici. Cet algorithme et ses optimisations feront l'objet
d'une explication soigneuse lors de la soutenance finale.


Enfin, Il est pr\'evu pour la suite de mettre au point un syst\`eme de
sauvegarde/charge des poids finaux des connexions afin de pouvoir
sauvegarder les valeurs du r\'eseau entra\^iné et les charger
directement une prochaine fois pour pouvoir utiliser ce dernier
imm\'ediatement sans avoir besoin d'effectuer l'apprentissage \`a chaque
utilisation, ce qui s'av\`ere \^etre plus qu'indispensable pour un OCR!



% subsection entrainement_du_reseau (end)

\subsection{Le banc de tests du perceptron} % (fold)
\label{subsec:le_banc_de_tests_du_perceptron}

Un petit shell a \'et\'e impl\'ement\'e afin de tester notre perceptron
au fur et \`a mesure de sa cr\'eation. Il est lanc\'e par d\'efaut lors
de l'ex\'ecution du programme ``network'' et propose diverses
options. 
%une grosse liste à puce please !
\begin{itemize}
	\item Trois fonctions d'affichage sont disponibles : 
\begin{description}
	\item[\verb!dispc! :] affiche l'\'etat des connexions neuronales et leur nombre
	\item[\verb!dispn! :] affiche l'\'etat des neurones et leur nombre
	\item[\verb!dispo! :] affiche les valeurs du vecteur de sorties (outputs)
\end{description}
%fin de liste, merci !

\item On peut initialiser le r\'eseau avec la fonction init, qui demandera
successivement de renseigner le nombre de couches cach\'ees d\'esir\'ees
ainsi que le nombre de neurones par couche cach\'ee.

\item La fonction train entra\^ine le r\'eseau (pour l'instant assez mal, il
faut l'admettre) 

\item La fonction active effectue une simple propagation des valeurs
d'entr\'ee vers celles de sortie

\item La fonction run permet (Bertrand) quant \`a elle, de pr\'eciser l'indice de la donn\'ee
que l'on souhaite tester (donn\'ee pr\'esente dans data.ml ; la donn\'ee
par d\'efaut est celle correspondant au XOR, il suffit de swaper les
noms de datatab et datatab2 dans data.ml pour acc\'eder au vecteur de
donn\'ees de l'afficheur sept segments !)

\item Enfin une fonction load et une fonction save sont pr\'esentes pour une
impl\'ementation future proche !

\end{itemize}

%fin de grosse liste à puce danke schön !

Ce shell de test sera am\'elior\'e et compl\'et\'e afin de rendre les
tests de notre r\'eseau rapides et efficaces. Il n'est pas destin\'e \`a
l'utilisateur final et c'est pourquoi nous avons pris quelques
libert\'es quant \`a son nombre de ligne de code.
On pourra bient\^ot pr\'eciser l'ensemble des param\`etres et
coefficients du r\'eseau via le shell, ce qui nous permettra
d'identifier clairement les options que nous t\^acherons,
,\'eventuellement dans une autre vie, de linker \`a une interface graphique
sp\'ecifique au r\'eseau qui sera enti\`erement param\'etrable et donc
r\'eutilisable \`a part enti`ere pour d'autres projets.
%subsection le_banc_de_tests_du_perceptron (end)

% section reseau_de_neurones (end)
\chapter{Site Web} % (fold)
\label{cha:site_web}
	Lors de la soutenance pr\'ec\'edente, nous avions pr\'esent\'e un site assez minimaliste. Tout en \textsc{Html}, peu de pages, vides pour la plupart, bref, on s'\'etait pas foul\'e. Beaucoup de choses ont chang\'ees. Tout d'abord, le site est rempli! On a une belle page ``About'' contenant un r\'esum\'e de nos vies passionantes et pleines d'aventures, un page de t\'el\'echargement contenant pour l'instant la \emph{tarball} rendue lors que la premi\`ere soutenance et son rapport.
	
	La grosse nouveaut\'e, c'est la page ``D\'eveloppement'', qui contient\ldots un script de news en PHP! Ce script s'est r\'ev\'el\'e tr\`es simple \`a \'ecrire. Nous avons eu besoin de 3 pages en php uniquement, ainsi qu'une base de donn\'ee MySQL. Une page \verb!liste_news.php! liste les news d\'ej\`a \'ecrites et permet de les \'editer ou de les supprimer. Une autre page \verb!rediger_news.php! nous permet de r\'ediger une nouvelle news. \\
	Voici à quoi ressemble notre site:
	
	
	\begin{center}

		\includegraphics[width=120mm]{screen_site.png}\\
		\caption{\emph{Calcul de la sortie d'un neurone}}\\
	\end{center}

	
% chapter site_web (end)

\chapter{Conclusion} % (fold)
\label{cha:conclusion}

Nous voil\`a arriv\'es au terme de cette deuxi\`eme soutenance ! L'hiver
a \'et\'e rude et nombreux sont les braves tomb\'es au combat. Mais la HUGE Team est
toujours l\`a, plus bouillante que jamais, bien d\'etermin\'ee \`a aller
jusqu'au bout dans sa qu\^ete du Saint-OCR. 
Le bilan est d'ailleurs somme toute assez encourageant : le plus gros du
pr\'etraitement a \'et\'e fait et l'interface graphique en est
d\'esormais \`a un stade bien avanc\'e. Le r\'eseau de neurones est
quasiment termin\'e et les premi\`eres s\'eries de tests sur alphabet
complet sont d\'ej\`a planifi\'ees. L'essentiel de ce qui nous attend
dans cette derni\`ere ligne droite consiste principalement en
l'optimisation de chacun des modules de notre OCR. Il nous faudra \'evidemment linker
chacun d'eux \`a notre interface graphique, et tester les performances
globales du programme. La documentation d'OCRe devra \'egalement \^etre
r\'edig\'ee avec soin, avec l'aide des outils OCamlDoc et Doxygen, que
nous avons d\'ej\`a commenc\'e \`a utiliser. Bref, la guilbertisation du
monde est en bonne marche, et quatre mots (+ 2 caract\`eres qui font 42) suffiront bien pour conclure
cet épisode : vivement notre prochaine entrevue ;)

% chapter Conclusion (end)


\appendix

\section{D\'etection des caract\`eres} % (fold)
\label{sec:d'etection_des_caract`eres}

\begin{center}
	
	\includegraphics[width=160mm]{iop.jpg}\\
\end{center}

% section d'etection_des_caract`eres (end)

\end{document}

%---FIN-DE-DOCUMENT------------------------------------------------------------
