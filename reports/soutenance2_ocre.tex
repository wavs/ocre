%---DOCUMENT-------------------------------------------------------------------

\documentclass[a4paper,10pt]{report} 
\usepackage[french]{babel} 
\usepackage[T1]{fontenc} 

%---PACKAGES-------------------------------------------------------------------

\usepackage{makeidx} \makeindex
\usepackage[Rejne]{fncychap}				% Lenny, Conny ,Bjarne, Rejne, Glenn, Sonny
\usepackage{fancyhdr}
\usepackage{eurosym}
\usepackage{lastpage}
\usepackage{a4wide}
\usepackage[french]{minitoc}
\usepackage[hmargin=1cm,vmargin=2cm]{geometry}

%---SORTIES--------------------------------------------------------------------

\newif\ifpdf

\ifx\pdfoutput\undefined
   \pdffalse
\else 
   \ifnum\pdfoutput=0
      \pdffalse
   \else
      \pdfoutput=1 \pdftrue
   \fi
\fi


%---PDF------------------------------------------------------------------------

\ifpdf
\usepackage[pdftex]{graphicx, color}
\graphicspath{{images/}}
\DeclareGraphicsExtensions{.jpg,.png}
\pdfcompresslevel=9

\usepackage[pdftex, 					% Param�trage de la navigation
bookmarks = true, 						% Signets
bookmarksnumbered = true, 		% Signets num�rot�s
pdfpagemode = None, 					% None, UseThumbs, UseOutlines, Fullscreen
pdfstartview = FitH, 					% FitH, FitV, FitR, FitB, FitBH, FitBV, Fit
pdfpagelayout = OneColumn, 		% SinglePage, OneColumn, TwoColumnLeft, TwoColumnRight
colorlinks = false, 					% Liens en couleur
urlcolor = black, 						% Couleur des liens externes
pdfborder = {0 0 0} 					% Style de bordure : ici, rien
]{hyperref}

\hypersetup{ 									
pdfauthor = {\textsc{Huge Software}\\ Thomas A\"it-Taleb, Dimitri Georgoulis, Pierre Guilbert et Alexandre Testu}, 							% Auteurs
pdftitle = {Rapport de soutenance}, 								% Titre du document
pdfsubject = {Deuxi\`eme Soutenance}, 							% Sujet
pdfkeywords = {}, 						% Mots-clefs
pdfcreator = {}, 							% Logiciel qui a cr�e le document
pdfproducer = {} 							% Soci�t� avec produit le logiciel
plainpages = false}
\usepackage{pdfpages}

%---DVI------------------------------------------------------------------------

\else
\usepackage{graphicx}
\graphicspath{{eps/}}
\newcommand{\url}[1]{\emph{#1}}
\newcommand{\href}[2]{\emph{#2}[1]}
\fi

%---EN-TETE-ET-PIED-DE-PAGE----------------------------------------------------

\renewcommand{\headrulewidth}{0.5pt}
\renewcommand{\footrulewidth}{0.5pt}
\pagestyle{fancy}

%\lhead{}
%\chead{}
%\rhead{}
%\lfoot{}
%\cfoot{}
%\rfoot{}

%---PAGE-DE-GARDE--------------------------------------------------------------

\title{\textsc{Rapport de Soutenance} \\ Deuxi\`eme Soutenance}
\author{\textsc{Huge Software}\\ \\ Thomas A\"it-Taleb \\ Dimitri Georgoulis \\ Pierre Guilbert \\ Alexandre Testu}
\date{}

%---COLOR---------------------------------------------------------------------

%\pagecolor{}
% \color{}

%---DEBUT-DU-DOCUMENT----------------------------------------------------------

\begin{document}
\dominitoc
\maketitle
\tableofcontents \pagebreak
\thispagestyle{fancy}

\chapter{Introduction} % (fold)
\label{cha:introduction}



% chapter introduction (end)

\chapter{Ce qu'on a fait} % (fold)
\label{cha:ce_qu_on_a_fait}

\section{Interface Utilisateur} % (fold)
\label{sec:interface_utilisateur}
Lors de la premi\`ere soutenance, nous pouvions ouvrir une image (\'enorme, hein?). C'est \`a peu pr\`es tout. 
Beaucoup de choses ont chang\'e depuis le mois de F\'evrier. 
Les modifications apport\'ees sont relatives au texte. Nous pouvons d\'esormais \'editer du texte, ainsi que l'enregistrer. Soyons franc: la gestion du texte sous Gtk, c'est pas de la tarte. On utilise un widget de type GtkTextView qui nous sert \`a afficher le texte. Pour l'\'editer, on utilise un GtkTextBuffer, qui utilise des GtkTextIter associ\'es \`a des GtkTextMarks. Bref, c'est un merdier sans nom.
Les GtkTextIter servent \`a se rep\'erer dans le GtkTextBuffer (qui est un buffer, thank you captain Obvious!). Un GtkTextIter est un endroit du texte. On l'utilise par exemple pour r\'ecup\'erer une cha\^ine de caract\`eres:
	\begin{verbatim}
		
	\end{verbatim}
% section interface_utilisateur (end)



% chapter ce_qu_on_a_fait (end)























\end{document}

%---FIN-DE-DOCUMENT------------------------------------------------------------
