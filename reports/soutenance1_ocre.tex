%
%  Rapport de soutenance 1 // OCRe - HUGE Software
%
%  Created by Alexandre Testu on 2008-02-01.
%  Copyright (c) 2008 Epita. All rights reserved.
%
\documentclass[]{report}

% Use utf-8 encoding for foreign characters
\usepackage[utf8]{inputenc} 
\usepackage[french]{babel}

% Setup for fullpage use
\usepackage{fullpage}

% Uncomment some of the following if you use the features
%
% Running Headers and footers
\usepackage{fancyhdr}
% Multipart figures
%\usepackage{subfigure}
% More symbols
%\usepackage{amsmath}
%\usepackage{amssymb}
%\usepackage{latexsym}
% Surround parts of graphics with box
\usepackage{boxedminipage}

% Package for including code in the document
\usepackage{listings}

% If you want to generate a toc for each chapter (use with book)
\usepackage{minitoc}

% This is now the recommended way for checking for PDFLaTeX:
\usepackage{ifpdf}

%\newif\ifpdf
%\ifx\pdfoutput\undefined
%\pdffalse % we are not running PDFLaTeX
%\else
%\pdfoutput=1 % we are running PDFLaTeX
%\pdftrue
%\fi
\ifpdf 
\usepackage[pdftex]{graphicx} \else 
\usepackage{graphicx} \fi 
\title{OCRe} 
\author{ \textsc{Huge} Software \\
\\
\\
Premi\`ere Soutenance \\
\\
\\
\textsc{Rapport}}

\date{}

\begin{document}

\ifpdf \DeclareGraphicsExtensions{.pdf, .jpg, .tif} \else \DeclareGraphicsExtensions{.eps, .jpg} \fi

\maketitle

\pagebreak

\tableofcontents 
\pagebreak

\part{Pr\'eparatifs} 

% (fold)
\label{prt:preparatifs}

\chapter{Gen\`ese} % (fold)
\label{cha:genese}
	\section{Naissance du groupe} % (fold)
	\label{sec:naissance_du_groupe} 
		Au commencement, il y avait quatre geeks. Trois rescap\'es de l'aventure Objectif:Mars trois ``poutreurs de l'espace'', c'est \`a dire Dimitri Georgoulis, Pierre Guilbert et Alexandre Testu ainsi qu'une nouvelle recrue, le sieur Thomas A\"it-taleb. Nous sommes tous des amis proches, et nous pensons que c'est ce qui nous diff\'erencie de beaucoup d'autres groupes: nous sommes soud\'es. Les diff\'erends qu'il peut y avoir au sein du groupe se r\`egle d'un claquement de doigt, et les choix que l'\'equipe a eu \`a faire ont toujours \'etait fait \`a l'unanimit\'e.
		\paragraph{\textsc{Huge}?, OCRe?} % (fold)
		\label{par:huge_ocre_}
			Pourquoi \textsc{Huge}? Nous m\^eme avons du mal \`a l'expliquer. L'expression remonte \`a l'ultime ``coding week'' de sup, durant laquelle ``Huge!'' est devenu une sorte de signe de ralliement. Tout n'\'etait que hugeitude \`a cette \'epoque.
			Pourquoi OCRe? Les trois premi\`eres lettres sont \'evidentes, et la troisi\`eme n'est autre que le ``e'' d'Epita. De plus, OCRe est plus facile \`a prononcer qu'``OCR''.
		% paragraph huge?_ocre_ (end)
	% section naissance_du_groupe (end)

	\section{Choix du sujet} % (fold)
	\label{sec:choix_du_sujet} 
		Le fait de choisir de r\'ealiser un OCR est apparu comme une \'evidence pour tous les membres du groupe. Nous avons d\'ej\`a fait de la 3D l'ann\'ee derni\`ere, et le fait de d\'evelopper un logiciel de \emph{reconaissance d'\'ecriture} nous fascine tous les quatre. 
	% section choix_du_sujet (end)
% chapter genese (end)

\chapter{Organisation} % (fold)
\label{cha:organisation}
	On entend par organisation du projet la r\'epartition des t\^aches, la communication entre les membres du groupe
	\section{Applications utilis\'ees} % (fold)
	\label{sec:applications_utilis'ees}
		\subsection{En ligne} % (fold)
		\label{sub:en_ligne}
			\paragraph{Basecamp\\} % (fold)
			\label{par:basecamp}
			Basecamp\footnote{\emph{http://www.basecamphq.com/}} est une application web de gestion de projet. Elle est développ\'e en Ruby on Rails par une start up bas\'ee \`a Chicago, 37Signals. Cette application est tr\`es intuitive, elle nous a beacoup aid\'e \`a collaborer de mani\`ere plus efficace que pendant le projet de Sup. Basecamp a \'et\'e recommand\'ee par le Wall Street Journal, Time, Business 2.0 ou encore BusinessWeek.
			% paragraph basecamp (end)
			\paragraph{Bubbl.us\\} % (fold)
			\label{par:bubbl_us}
			Il s'agit d'un logiciel de \emph{brainstorming} en Flash. Nous l'avons utilis\'e pour mieux visualiser la strusture qu'allait prendre notre OCR. La structure obtenue est visible en annexe. %ANNEXE
			% paragraph bubbl_us (end)
			\paragraph{Wordpress\\} % (fold)
			\label{par:wordpress}
			Nous utilisons la plateforme de publication Open Source Wordpress pour notre site web. En effet, nous aurions tr\`es bien pu passer du temps \`a d\'evelopper nos propre script PHP. Mais jamais ils ne serait jamais arriv\'e \`a la cheville de Wordpress. Exit le PHP donc, nous n'aurons qu'\`a toucher \`a du CSS et du HTML.
			% paragraph wordpress (end)
			% subsection en_ligne (end)
		\subsection{Hors ligne} % (fold)
		\label{sub:hors_ligne}
			\paragraph{Emacs\\} % (fold)
			\label{par:emacs}
				C'\'etait soit \c ca soit Vi\ldots
			% paragraph emacs (end)
			\paragraph{TextMate\\} % (fold)
			\label{par:textmate}
				TextMate est le meilleur \'editeur de texte disponible sur Mac (oui, un troll se cache dans cette phrase). Plus s\'erieusement, le Mac \emph{user} du groupe, Alexandre Testu, celui qui r\'edige ce rapport \`a la troisi\`eme personne appr\'ecie l'auto-compl\'etion en \LaTeX qu'offre TextMate.
			% paragraph textmate (end)
			\paragraph{Subversion\\} % (fold)
			\label{par:subversion}
				Nous avons enregistr\'e OCRe sur Google Code\footnote{\emph{http://code.google.com/p/ocre}}. Pourquoi pas Sourceforge ou autre? Google Code offre tout d'abord une interface bien plus agr\'eable, il n'y a pas de publicit\'es qui prennent un tiers de la page comme sur Sourceforge. De plus, nous avons un Wiki \`a notre disposition. Google Code nous est donc apparu comme le meilleur choix pour nous comme pour nos futurs utilisateurs.
			% paragraph subversion (end)
			\paragraph{Glade} % (fold)
				Glade est un logiciel qui facilite le d\'eveloppement d'interface en Gtk. Pour plus de d\'etail, se r\'ef\'erer \`a %REF
			\label{par:glade}
			% paragraph glade (end)
		% subsection hors_ligne (end)
	% section applications_utilis'ees (end)
% chapter organisation (end)

% part preparatifs (end)




\part{Ce qu'on a fait} % (fold)
\label{prt:ce_qu_on_a_fait}


	\chapter{Extraction} % (fold)
	\label{cha:extraction}
	
		%%%%%%%%%%%%%%%%%%%%%%%%%%%%%%%%%%%%%%%%%%%%
		%%%%%%%%%%%%%%%%%%%%%% FIXME
		%%%%%%%%%%%%%%%%%%%%%%%%%%%%%%%%%%%%%%%%%%%%
		%%%%%%%%%%%%%%%%%%%%%% THOMAS
		%%%%%%%%%%%%%%%%%%%%%%%%%%%%%%%%%%%%%%%%%%%%
	
	% chapter extraction (end)


	\chapter{Pr\'etraitement} % (fold)
	\label{cha:pr'etraitement}
	
		%%%%%%%%%%%%%%%%%%%%%%%%%%%%%%%%%%%%%%%%%%%%
		%%%%%%%%%%%%%%%%%%%%%% FIXME
		%%%%%%%%%%%%%%%%%%%%%%%%%%%%%%%%%%%%%%%%%%%%
		%%%%%%%%%%%%%%%%%%%%%% PIERRE
		%%%%%%%%%%%%%%%%%%%%%%%%%%%%%%%%%%%%%%%%%%%%
		
	% chapter pr'etraitement (end)

	
	\chapter{R\'eseau de neurones} % (fold)
	\label{cha:r'eseau_de_neurones}
	
		%%%%%%%%%%%%%%%%%%%%%%%%%%%%%%%%%%%%%%%%%%%%
		%%%%%%%%%%%%%%%%%%%%%% FIXME
		%%%%%%%%%%%%%%%%%%%%%%%%%%%%%%%%%%%%%%%%%%%%
		%%%%%%%%%%%%%%%%%%%%%% DIMITRI
		%%%%%%%%%%%%%%%%%%%%%%%%%%%%%%%%%%%%%%%%%%%%
		
	% chapter r'eseau_de_neurones (end)


	\chapter{Interface utilisateur} % (fold)
	\label{cha:interface_utilisateur}
	
		%%%%%%%%%%%%%%%%%%%%%%%%%%%%%%%%%%%%%%%%%%%%
		%%%%%%%%%%%%%%%%%%%%%% FIXME
		%%%%%%%%%%%%%%%%%%%%%%%%%%%%%%%%%%%%%%%%%%%%
		%%%%%%%%%%%%%%%%%%%%%% ALEX
		%%%%%%%%%%%%%%%%%%%%%%%%%%%%%%%%%%%%%%%%%%%%
		
	% chapter interface_utilisateur (end)
	
	\chapter{Site Web} % (fold)
	\label{cha:site_web}
		\section{Ce qu'on veut} % (fold)
		\label{sec:ce_qu_on_veut}
		Le site web\footnote{\emph{http://huge.ocre.free.fr}} est la premi\`ere chose que verra l'utilisateur potentiel. Il devra donc \^etre \`a la fois attirant et bien ordonn\'e. Le site devra comporter:
		\begin{itemize}
			\item Une page d'accueil qui expliquera bri\`evement ce qu'est notre projet, qui nous sommes, etc\ldots
			\item Une page de d\'eveloppement qui contiendra un lien vers notre page Google Code.
			\item Une page de t\'el\'echargement qui listera chronologiquement tout ce qu'on a cr\'e\'e, c'est \`a dire nos sources, les \'ex\'ecutables et les rapports de soutenances.
			\item Une partie ``blog'' qui informera le visiteurs du stade de d\'eveloppement du projet.
			\item Une page d'``A propos'' qui expliquera plus en d\'etail qui nous somme, ce qu'est l'Epita, etc\ldots
		\end{itemize} 
		% section ce_qu_on_veut (end)
		\section{Comment on s'est d\'ebrouill\'e} % (fold)
		\label{sec:comment_on_s_est_d'ebrouill'e}
		Comment on s'est débrouillé

			On est parti avec l'idée que le site web ne devrait pas nous éloigner de notre tâche principale: développer un logiciel de reconnaissance de caractères. Il fallait donc trouver un moyen rapide pour construire ce site. Nous nous sommes donc tournés vers la plateforme de publication open source Wordpress.
			Nous avons pleinement conscience que ce choix nous empêchera d'apprendre en détail des langages tels que le PHP. Cependant, entre passer plus de temps à peaufiner notre OCR et coder un script de news, il n'y a pas photo!

			Wordpress s'installe très facilement sur notre site. Un glisser-déposer et 3 clics plus tard, notre site fonctionne. Par contre, il ressemble à n'importe quel blog, pas du tout au site d'une boite de logiciel. Qu'à cela ne tienne, Alexandre sort CSSEdit\footnote{http://macrabbit.com/cssedit/} de son dossier Applications, et quelques heures plus tard\ldots tadam\footnote{http://huge.ocre.free.fr/}!

		Finalement, même si nos connaissances du PHP se limite toujours à\ldots pas grand chose, on a construit un site ergonomique, tout en apprenant un autre langage, le CSS.

		% section comment_on_s_est_d'ebrouill'e (end)
	% chapter site_web (end)
% part ce_qu_on_a_fait (end)



\part{Ce qu'on compte faire} % (fold)
\label{prt:ce_qu_on_compte_faire}

% part ce_qu_on_compte_faire (end)

















\bibliographystyle{plain} 
\bibliography{} \end{document} 
