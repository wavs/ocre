%% debut du chapitre sur le pretraitement
\chapter{ OCRed }
%%mettre ca en jolie pour TESTU
par guilbep et dimok
%%mettre ca en jolie pour TESTU
\section{Le pretraitement}

\section{Son utilisation dans OCRe}
\subsection{}
%% fin du chapitre sur le pretraitement

%% debut du chapitre sur les reseaux de neuronnes
\chapter{ R\'eseau de Neurones Formels?}
%%mettre ca en jolie pour TESTU
fait par guilbep et dimok
%%mettre ca en jolie pour TESTU

\section{ Qu'est-ce qu'un RNF?}
\subsection{Historique}
\subsection{Ca sert a quoi?}
\subsection{Diff\'erent Mod\`ele}
\subsection{Exemple}

\section{RNF et OCRe}
\subsection{Quel mod\`ele a-t-on choisi?}
\subsection{Quelle structure pour notre RNF?}
\subsection{Exp\'eriences et rat\'es}
\subsection{Conclusion}

%% fin du chapitre sur les reseaux de neuronnes
%%
%% + Reprise du cahier des charges,
%% + Plusieurs presentations possibles :
%%      - Chronologique (groupe),
%%      - Chronologique (individuelle),
%%      - Individuelles (repartition des taches),
%%      - Autres.
%% + Recit de la realisation :
%%      - Ses joies,
%%      - Ses peines,
%%      - Etc.
%% + Les annexes comprennent :
%%      - Les exemples d'impression,
%%      - les exemples d'       cran,
%%      - les jeux d'essai,
%%      - les dessins d'origine,
%%      - Etc.t
