%% debut du chapitre sur le pretraitement
\chapter{ OCRed }
%%mettre ca en jolie pour TESTU
par guilbep et dimok
%%mettre ca en jolie pour TESTU
\section{Le pretraitement}

\section{Son utilisation dans OCRe}
\subsection{}
%% fin du chapitre sur le pretraitement

%% debut du chapitre sur les reseaux de neuronnes
\chapter{ R\'eseau de Neurones Formels?}
%%mettre ca en jolie pour TESTU
fait par guilbep et dimok
%%mettre ca en jolie pour TESTU

\section{ Qu'est-ce qu'un RNF?}
Un RNF est un r\'eseau de neurones form\'els, bon jusque l\'a vous me
suivez, quoique pas sur. On va essayer d'y remedier.
\subsection{Historique}
Le domaine des r\'eseaux de neurones, dont les traditions sont
anciennes, a \'et\'e saisi par une p\'eriode d'effervescence
the\'eorique et interdisciplianire pendant les ann\'ees quatre-vingt,
suivie par une phase d'explorations tous azimuts au cours de la
d\'ecennie suivant. Ca c'est pour l'introduction.
Pendant les ann\'ees 1980, les travaux et recherches sur les neurones
formels rassemblaient une multitude de chercheurs venant d'horizons
tout aussi divers que la physique statistique, la biologie,
la psychologie et l'ing\'enierie, qui esp\'erer profiter de cette
situation.

Collaborations entre biologistes et physiciens se sont prolong\'e
surtout sur des probl\`emes th\'eoriques dans les ann\'ees 90. A cette
m\'eme \'epoque, les ing\'enieurs ont d'ailleurs commenc\'es la
r\'ealisation gr\^ace aux rnfs, de syst\'emes fonctionnels en
reconnaissance de formes et en mod\'elisation non-lin\'eaire.
Ces derni\`eres avanc\'ees sont notamment du aux r\'esultats
math\'ematiques et statistiques importants qui on vu le jour en ce
d\'ebut de d\'ecennie(1990 j'entends). De plus la
recherches dans ce domaine, qui n'a pas cess\'e, a apport\'e de nombreuses solutions
d'optimisation ces derni\'eres ann\'ees, principalement pour les
prob\'emes d'estimations des param\'etres du reseaux, et des
m\'ethodes d'\'evaluation de la qualit\'e d'un classifieur.

Rappelons, qu'au d\'epart l'on a appell\'e cela par analogie avec les
syst\'emes biologiques, car oui nous sommes tous dot\'ees plus ou
moins d'un certain nombres de neurones. M\^eme moi.

Aujourd'hui principalement \'a cause de la nature statistiques des
r\'eseaux de neurones, cette analogie semble devenir d\'esu\`ete.
On pourrait m\^eme dire que les pseudo-justifications biologiques des
architectures et des algorithmes ne sont plus de mise, mais l\`a ce
n'est pas de moi.

Et l\`a normalement vous me dites, et avant les ann\'ees quatre-vingt
y'avait quoi? Bonne question: \dots d'apr\`es mes sources, une certaine
personne du nom de Marvin Minsky, connu pour ses th\'eories sur la
hierarchisation du syst\'eme de pens\'ees humain, mais aussi pour sa
d\'ecouverte de l'inutilit\'e du r\'eseau de neurones formels tels
qu'il \'etait envisag\'e dans les ann\'ees 70. Rummelhart introduit
alors le perceptrons multicouches en 1984, dont la grande nouveaut\'e
est la r\'etropropagation du gradient.

\subsection{Ca sert a quoi?}
\subsection{Diff\'erent Mod\`ele}
\subsection{Exemple}

\section{RNF et OCRe}
\subsection{Quel mod\`ele a-t-on choisi?}
\subsection{Quelle structure pour notre RNF?}
\subsection{Exp\'eriences et rat\'es}
\subsection{Conclusion}

%% fin du chapitre sur les reseaux de neuronnes
%%
%% + Reprise du cahier des charges,
%% + Plusieurs presentations possibles :
%%      - Chronologique (groupe),
%%      - Chronologique (individuelle),
%%      - Individuelles (repartition des taches),
%%      - Autres.
%% + Recit de la realisation :
%%      - Ses joies,
%%      - Ses peines,
%%      - Etc.
%% + Les annexes comprennent :
%%      - Les exemples d'impression,
%%      - les exemples d'       cran,
%%      - les jeux d'essai,
%%      - les dessins d'origine,
%%      - Etc.t
