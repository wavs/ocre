\chapter{Utilitaire de segmentation: seg}

\section{Introduction}
  L'�xecutable \emph{seg} permet de segmenter une image
pr�trait�e. Il fournit une image de pr�visualisation
des blocs segment�s ainsi qu'un fichier xml contenant
la structure du document et les donn�es.

\section{Installation de seg}
  \begin{enumerate}
    \item Placez vous dans le r�pertoire principal de l'OCR.
    \item Tapez la commande \verb!make! dans votre terminal.
    \item Patientez durant l'installation.
    \item L'installation est finie. Elle a cr�� un r�pertoire
          \verb!/bin! contenant l'�xecutable \emph{seg}.
  \end{enumerate}
  
P.S.: La commande \verb!make! permet l'installation de tous les
utilitaires pr�sents dans l'OCR.

\section{D�sinstallation de seg}
  \begin{enumerate}
    \item Placez vous dans le r�pertoire principal de l'OCR.
    \item Tapez la commande \verb!make clean! dans votre terminal.
    \item Patientez durant la d�sinstallation.
    \item Pour finir la d�sinstallation, tapez la commande
          \verb!make cleanc! dans votre terminal.
    \item La d�sinstallation est finie. Elle a supprim� le
          r�pertoire \verb!/bin! .
  \end{enumerate}

\section{Fonctionnement de seg}
  L'utilitaire \emph{seg} a pour but d'analyser une image pr�trait�e et
  d'en extraire le texte. Le programme g�re des fichiers images de type
  bitmap (bmp). Cette �xecutable prend en param�tre plusieurs options:
    \begin{itemize}
      \item -i [file] : permet de sp�cifier le fichier bitmap que le
            programme devra analyser.
      \item -o [file] : permet de sp�cifier le nom du fichier bitmap de
            pr�visualisation que produira le programme.
      \item -v : permet d'afficher les informations du processus
            r�alis� par le programme (verbose).
      \item -h : permet d'afficher l'aide de l'utilitaire.
    \end{itemize}
  
  Voici un exemple d'utilisation du programme:
    \begin{enumerate}
      \item Placez vous dans le r�pertoire principal de l'OCR.
      \item Changez de r�pertoire en tapant la commande
            \verb!cd /bin!
      \item Tapez ensuite la commande \verb!./seg -i myfile.bmp -o preview.bmp!
            o� myfile.bmp est le fichier � analyser.
      \item Le programme s'�xecute. Si vous avez rajout� l'option -v
            (verbose) vous verez s'afficher les �tapes r�alis�es par le
            programme.
      \item Dans ce cas-ci, le programme a analys� l'image \verb!myfile.bmp!
            et � g�n�rer dans le r�pertoire \verb!/bin! le fichier \verb!preview.bmp!
            ainsi que le fichier \verb!seg_report.xml! utile pour les autres
            utilitaires.
    \end{enumerate}

\section{Conseils d'utilisation}
  Nous vous conseillons d'utiliser \emph{seg} avec des images produites
par l'utilitaire de pr�traitement \emph{OCRed}, g�n�r�es par exemple par
cette ligne de commande � taper dans votre terminal au niveau du
r�pertoire \verb!/bin!: \verb!./OCRed -i myimage.jpg -o myfile.bmp -s 150!