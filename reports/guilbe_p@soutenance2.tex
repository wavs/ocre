%%debut chapitre sur le programme ocred
\chapter{ OCRed qu'est-ce que c'est ? }
 OCRed est n\'e des mots Ocre et digital, qui rappellons le encore une fois
 signifie ``Optical Character Recognition enhanced'' pour l'un et
 num\'erique pour l'autre. La partie du projet portant sur le
 pr\'etraitement de l'image devant \^etre faite en ocaml, OCRed est
 totalement r\'ealis\'e dans ce langage.

\section{ Visualisation de l'image grace a l'executable }
 La cr\'eation du logiciel de traitement de l'image nous a men\'e, au
 d\'ebut du projet \'a vouloir visualiser directement, c'est \'a dire
 sans logiciel tiers, l'action de notre travail.
 La visualisation des images trait\'ees permet notre ind\'ependances
 au niveau de l'utilisation du logiciel, ce qui est un confort certain.
 La fenetre est r\'eali\'ee avec OcamlSDL, tout comme le pr\'etraitement
 de l'image.
%% mettre image de la fenetre sdl
%% visualisation.jpg

\section{ Utilisation de l'executable }
 L'executable en ligne de commande admet un certain nombre d'arguments,
 certain ne pouvant evidemment donner lieu \'a un quelconque r\'esultat
 sans la sp\'ecification d'une image d'entr\'ee.
\subsection{ Sp\'ecifier une image d'entr\'ee }
 Il est plus logique, pour un logiciel de traitement d'image, d'avoir
 une image \'a traiter. Vous devez pour cela utilisez l'argument ``-i''
 suivi du chemin d'une image de type jpeg,bmp,png au moins.
\subsection{ Sp\'ecifier une image de sortie }
 Vous n'\'etes pas oblig\'e de pr\'eciser le chemin de l'image de sorties
 mais cela permet une certaine flexibilit\'ee. Dans le cas o\'u vous
 n'indiquez pas la sortie, OCRed se debrouille pour sortir une image
 avec un nom assez \'evident, d\'ependant du traitement. Par exemple
 rotatation.bmp ou encore tresholded.bmp, les images se trouveront l\'a
 o\'u vous avez execut\'e votre programme. Pour specifier un chemin de
 sortie veuillez utiliser l'argument ``-o'' suivi du chemin d'une image
de type jpeg,bmp,png au moins.
\subsection{ La visualisation de l'imaage }
 Gr\'ace \'a OcamlSDL, nous pouvons visualiser notre image, il faut pour
 cela specifier une image d'entr\'ee, et passer l'argument ``-d'' en
 parametre. Les d\'esavantage de l'utilisation de cette interface sont
 assez subtiles, par exemple vous ne pouvez ni zoomer ni dezoomer, ni
 vous deplacer sur l'image. Toutefois il y'a des avantages, vous pouvez
 visualiser directement les modifications appliqu\'ees \'a votre
 image. En appuyant sur F2 vous faites tourner l'image de l'angle que
 vous avez passer en param\'etre --voir plus loin--. En appuyant sur F3
 vous appliquez le seuil, que vous pouvez sp\'ecifier en ligne de
 commande. En appuyant sur F4, vous appliquez le filtre m\'edian. Nous
 tenons \'a vous rappeller que le seuil ne marche que sur les couleurs.
 Pour quitter il suffit d'appuyer sur F1, la petit croix en haut \'a
 droite ne marche pas.
\subsection{ Le seuil }
 Permet le passage d'une image en noir et blanc, pour avoir un minimum
 de pertes d'informations vous devez specifiez un seuil viable. Par
 d\'efaut le seuil et \'a 200, ce qui est suffisant pour un contraste
 normal de l'image. Vous devrez l'augmenter si il est trop faible et le
 baisser dans le cas contraire. Passer l'argument ``-s'' en
 parametre si vous voulez changez la valeur par d\'efaut. Pour
 l'anecdote ``-s'' viendrez de seuil, mais nous n'en sommes pas vraiment
 sur.
\subsection{ Rotation }
 Vous pouvez passez un angle en pr\'ecision r\'eel, ou enti\'ere. Pour
 cela passez respectivement l'angle avec l'argument ``-af'' et ``-a''.
 Rappelons que l'angle passez en parametres et en degr\'e et non en
 radian. Vous pouvez aussi tourner l'image de 90 degr\'e dans le sens
 direct ou indirect, gr\'ace \'a l'option ``-right'' et ``-left''.
\subsection{ Redimensionnement }
 Le redimensionnement peut s'effectuer en pourcentage de l'image avec
 l'option ``--resizepercent'' qui prend en param\'etre un pourcentage
 entier. L'on peut aussi redimensionner l'image en valeur absolue
 --c'est \'a dire en sp\'ecifiant des valeurs de hauteurs et de largeurs
 arbitraire--. Alexandre Testu avait besoin d'un petit redimensionnement
 rien que pour lui, c'est pour cela que l'option ``--resize-auto''
 existe.
\subsection{ detection d'angle}
 Il existe un argument magique ``-dev'' qui a juste besoin d'avoir une
 image qui n'est pas bien droite en param\'etre. Il retourne directement
 l'image dans le bon sens, et en noir et blanc.
\subsection{ Help me if you can! }
 L'option ``-help'' est aussi pr\'esente afin d'accompagner
 l'utilisateur dans sa d\'ecouverte du programme. Pour tout autre
 informations, il se peut que vous puissiez trouver des informations
 plus ou moins utiles dans les sources du projet, au niveau du fichier
 README.
%%fin chapitre sur le programme ocred


%%debut chapitre detection de l'angle
\chapter{ La d\'etection de l'angle }
 La d\'etection de l'angle \'etait \'a faire pour la prem\'ere
 soutenance, malheureusement par un manque de temps certain elle ne fut
 pas pr\'ete \'a temps. Nous arrivons \'a la deuxi\'eme soutenance avec
 deux types de d\'etection de l'angle de rotation du texte d'ume image.
 Ce qui a pour but de redresser l'image, afin d'avoir une segmentation
 \'a peur pr\'es correct.

\section{ Des principes assez diff\'erents}
 La d\'etection de l'angle de rotation d'un texte dans une image est un
 probl\'eme assez complexe. En effet une image contenant du texte n'est
 pas compos\'e de composante continues, il est donc plus difficile d'isoler
 les caract\'eristiques des \'el\`ements. Deux principes majeur nous
 sont apparu. Tout d'abord l'on \'etudie les caract\'eristiques d'une
 image pour chaque angle, l'on isole ensuite le bon angle, ce
 processus est long, d'autre part il est encore assez impr\'ecis.
 L'autre principe \'etait de d\'etecter la pente d'une ligne de caract\`ere.
\subsection{ La multiple rotation }
 Dans cet algorithme, l'on reduit la taille de l'image pour gagner en
 temps d'execution, le redimensionnement est calcul\'e en fonction de la
 taille d'origine de l'image, on suppose avoir du format A4 \'a chaque
 fois.

 Comme vous pouvez le voir sur ces images il est \'evident qu'il y'a une
 net diff\'erence entre des images avec un texte mal orient\'e. Ces
 images repr\'esente la projection horizontale des pixels noirs d'une image.
%% histogramme5degree.jpg
%% sur cette image l'on peut voir un histogramme d'une image avec un
 %% texte ayant subit une rotation de 5 degr\'e


%% rotation1deghisto.jpg
%% sur cette image l'on peut voir un histogramme d'une image avec un
 %% texte ayant subit une rotation de 1 degr\'e


%% histogramme_bon.jpg
%% sur cette image l'on peut voir un histogramme d'une image avec un
%% texte n'ayant pas subit de rotation.

\subsection{ Detection de la pente }
On detecte la pente d'une ligne de texte, pour cela l'on parcour l'imagepour
chaque ligne de pixels jusqu'a ce que l'on tombe sur un pixel noir.
% mettre l'image histo_pente.jpg

\section{ Optimisations des algorithmes }
\subsection{ Reduction de l'image }

%% reduceimage20percent75dpit.jpg

\subsection{ Quels limites et comment les surpasser ? }


%% chapter other problemes
\chapter{ Probl\'emes li\'ees au traitement de l'image }
\section{ Rotation de l'image}
\section{ Filtres diminuant la perte d'information}

%% OCRed qu'est-ce que c'est pourquoi c'est faire.

%% = l'executable
%% = la ligne de commande
%%  les differentes options toussa


%% La detection de l'angle de rotation
%%  idees d'optimisation
%%  redimensionnement de l'image
%%  mettre des images
%%  histogramme et traitement du signal
%%  \mettre des images
%%  multiple rotation
%%  etudes de la pente.
%%  mettre des images
%% detection d'angle mettre des images avant apres

%% La rotation d'une image, une idee qui ne marche pas

%% Les idees qui n'ont pas vu le jour
