\chapter{ OCRed qu'est-ce que c'est ? }
 OCRed est n\'e des mots Ocre et digital, qui rappellons le encore une fois
 signifie ``Optical Character Recognition enhanced'' pour l'un et
 num\'erique pour l'autre. La partie du projet portant sur le
 pr\'etraitement de l'image devant \^etre faite en ocaml, OCRed est
 totalement r\'ealis\'e dans ce langage.
\section{ Visualisation de l'image grace a l'executable }
 La cr\'eation du logiciel de traitement de l'image nous a men\'e, au
 d\'ebut du projet \'a vouloir visualiser directement, c'est \'a dire
 sans logiciel tiers, l'action de notre travail.
 La visualisation des images trait\'ees permet notre ind\'ependances
 au niveau de l'utilisation du logiciel, ce qui est un confort certain.
 La fenetre est r\'eali\'ee avec OcamlSDL, tout comme le pr\'etraitement
 de l'image.
%% mettre image de la fenetre sdl
\section{ Utilisation de l'executable }
 L'executable en ligne de commande admet un certain nombre d'arguments,
 certain ne pouvant evidemment donner lieu \'a un quelconque r\'esultat
 sans la sp\'ecification d'une image d'entr\'ee.
\subsection{ Sp\'ecifier une image d'entr\'ee }
 Il est plus logique, pour un logiciel de traitement d'image, d'avoir
 une image \'a traiter. Vous devez pour cela utilisez l'argument ``-i''
 suivi du chemin d'une image de type jpeg,bmp,png au moins.
\subsection{ Sp\'ecifier une image de sortie }
 Vous n'\'etes pas oblig\'e de pr\'eciser le chemin de l'image de sorties
 mais cela permet une certaine flexibilit\'ee. Dans le cas o\'u vous
 n'indiquez pas la sortie, OCRed se debrouille pour sortir une image
 avec un nom assez \'evident, d\'ependant du traitement. Par exemple
 rotatation.bmp ou encore tresholded.bmp, les images se trouveront l\'a
 o\'u vous avez execut\'e votre programme. Pour specifier un chemin de
 sortie veuillez utiliser l'argument ``-o'' suivi du chemin d'une image
de type jpeg,bmp,png au moins.
\subsection{ La visualisation de l'imaage }
 Gr\'ace \'a OcamlSDL, nous pouvons visualiser notre image, il faut pour
 cela specifier une image d'entr\'ee, et passer l'argument ``-d'' en
 parametre. Les d\'esavantage de l'utilisation de cette interface sont
 assez subtiles, par exemple vous ne pouvez ni zoomer ni dezoomer, ni
 vous deplacer sur l'image. Toutefois il y'a des avantages, vous pouvez
 visualiser directement les modifications appliqu\'ees \'a votre
 image. En appuyant sur F2 vous faites tourner l'image de l'angle que
 vous avez passer en param\'etre --voir plus loin--. En appuyant sur F3
 vous appliquez le seuil, que vous pouvez sp\'ecifier en ligne de
 commande. En appuyant sur F4, vous appliquez le filtre m\'edian. Nous
 tenons \'a vous rappeller que le seuil ne marche que sur les couleurs.
 Pour quitter il suffit d'appuyer sur F1, la petit croix en haut \'a
 droite ne marche pas.
\subsection{ Le seuil }
 Permet le passage d'une image en noir et blanc, pour avoir un minimum
 de pertes d'informations vous devez specifiez un seuil viable. Par
 d\'efaut le seuil et \'a 200, ce qui est suffisant pour un contraste
 normal de l'image. Vous devrez l'augmenter si il est trop faible et le
 baisser dans le cas contraire. Passer l'argument ``-s'' en
 parametre si vous voulez changez la valeur par d\'efaut. Pour
 l'anecdote ``-s'' viendrez de seuil, mais nous n'en sommes pas vraiment
 sur.
\subsection{ Rotation }
 Vous pouvez passez un angle en pr\'ecision r\'eel, ou enti\'ere. Pour
 cela passez respectivement l'angle avec l'argument ``-af'' et ``-a''.
 Rappelons que l'angle passez en parametres et en degr\'e et non en
 radian.
\subsection{ right }
\chapter{ La d\'etection de l'angle }
\section{ Des principes assez diff\'erents}
\subsection{ La multiple rotation }
\subsection{ Detection de la pente }
\section{ Optimisations des algorithmes }
\subsection{ Reduction de l'image }
\subsection{ Quels limites et comment les surpasser ? }
\chapter{ Probl\'emes li\'ees au traitement de l'image }
\section{ Rotation de l'image}
\section{ Filtres diminuant la perte d'information}

%% OCRed qu'est-ce que c'est pourquoi c'est faire.

%% = l'executable
%% = la ligne de commande
%%  les differentes options toussa


%% La detection de l'angle de rotation
%%  idees d'optimisation
%%  redimensionnement de l'image
%%  mettre des images
%%  histogramme et traitement du signal
%%  \mettre des images
%%  multiple rotation
%%  etudes de la pente.
%%  mettre des images
%% detection d'angle mettre des images avant apres

%% La rotation d'une image, une idee qui ne marche pas

%% Les idees qui n'ont pas vu le jour
