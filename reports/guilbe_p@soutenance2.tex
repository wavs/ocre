\chapter{ OCRed qu'est-ce que c'est ? }
 OCRed est n\'e des mots Ocre et digital, qui rappellons le encore une fois
 signifie \" Optical Character Recognition enhanced \" pour l'un et
 num\'erique pour l'autre. La partie du projet portant sur le
 pr\'etraitement de l'image devant \^etre faite en ocaml, OCRed est
 totalement r\'ealis\'e dans ce langage.
\section{ Visualisation de l'image grace a l'executable }
 La cr\'eation du logiciel de traitement de l'image nous a men\'e au
 d\'ebut du projet \'a vouloir visualiser directement, c'est \'a dire
 sans logiciel tiers, l'action de notre travail.
 La visualisation des images trait\'ees permet notre ind\'ependances
 au niveau de l'utilisation du logiciel.
\section{ Utilisation de l'executable }
\chapter{ La d\'etection de l'angle }
\section{ Des principes assez diff\'erents}
\subsection{ La multiple rotation }
\subsection{ Detection de la pente }
\section{ Optimisations des algorithmes }
\subsection{ Reduction de l'image }
\subsection{ Quels limites et comment les surpasser ? }
\chapter{ Probl\'emes li\'ees au traitement de l'image }
\section{ Rotation de l'image}
\section{ Filtres diminuant la perte d'information}

%% OCRed qu'est-ce que c'est pourquoi c'est faire.

%% = l'executable
%% = la ligne de commande
%%  les differentes options toussa


%% La detection de l'angle de rotation
%%  idees d'optimisation
%%  redimensionnement de l'image
%%  mettre des images
%%  histogramme et traitement du signal
%%  \mettre des images
%%  multiple rotation
%%  etudes de la pente.
%%  mettre des images
%% detection d'angle mettre des images avant apres

%% La rotation d'une image, une idee qui ne marche pas

%% Les idees qui n'ont pas vu le jour
